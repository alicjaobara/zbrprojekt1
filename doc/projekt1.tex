\documentclass[a4paper]{article}
\usepackage[utf8]{inputenc}
\usepackage[english,polish]{babel}
\usepackage{polski}
\usepackage[T1]{fontenc}
\frenchspacing
%\usepackage{indentfirst} % wciecia w akapitach po \\
\usepackage{geometry}
\geometry{left=2.5cm,right=2.5cm,%
	bindingoffset=0mm, top=2.5cm, bottom=2.5cm}

\usepackage{enumerate} %pakiet aby wybierać oznaczenia w wypunktowaniu enumerate


\begin{document}
	\renewcommand{\*}{\cdot}
	\renewcommand{\phi}{\varphi}
	\begin{titlepage}
		\author{Alicja Obara}
		\title{Projekt 1}
		\maketitle
	\end{titlepage}
	
	\section{Założenia projektu}
	Celem projektu było stworzenie wizualizacji ruchu punktu TCP podanego mechanizmu robota o strukturze kinematycznej {CR, BR1, BR2, BL, CL, AL }, budowie i parametrach geometrycznych omówionych na wykładzie.
	\begin{enumerate}
		\item wybór wartości parametrów geometrycznych części regionalnej mechanizmu: 
		\begin{enumerate}
			\item l1 [mm],
			\item l2 [mm],
			\item l3 [mm],
			\item d [mm],
			\item e [mm],
		\end{enumerate}
		\item wybór wartości parametrów geometrycznych części lokalnej mechanizmu:
		\begin{enumerate}
			\item l4 [mm],
			\item l5 [mm],
			\item l6 [mm],
		\end{enumerate}
		\item ustawienie wektora podejścia członów części lokalnej mechanizmu:
		\begin{enumerate}
			\item $\theta$ [stpn],
			\item $\psi$ [stpn],
		\end{enumerate}
		\item wybór punktu startowego i końcowego trajektorii ruchu z zadaną liczbą równooddalonych punktów podporowych przejścia,
		\begin{enumerate}
			\item punkt startowy,
			\item punkt końcowy,
			\item liczba kroków,
		\end{enumerate}
		\item wybór punktu startowego i końcowego trajektorii ruchu z żądaniem interpolowanego liniowo lub funkcyjnie przejścia pomiędzy zadanymi punktami,
		\item sprawdzenie warunków bezpiecznego przejścia pomiędzy zadanymi punktami,
		\item podanie przebiegu współrzędnych maszynowych-wartości zadanych sterowników-regulatorów położenia, wymuszających zadane przejście: $\phi_1, \phi_2, \phi_3, \phi_4, \phi_5$
		\item wprowadzenie kryteriów wyboru sposobu realizacji ruchu-przemieszczeń członów mechanizmu po zadanej trajektorii: $\delta_1, \delta_2, \delta_3$
	\end{enumerate}
	\section{Obliczenia}
	Przyjmuję oznaczenia: $St=\sin(\theta), Ct=\cos(\theta), Sp=\sin(\psi), Cp=\cos(\psi)$ \\
	 oraz $S1=\sin(\phi1), C1=\cos(\phi1), S12=\sin(\phi1+\phi2)$
	
	
	$x_P=x_T-l\*Ct\*Cp$
	$y_P=y_T-l\*Ct\*Sp$
	$z_P=z_T-l\*St$
	
	
	\section{Wykresy}
	
	\section{Kod}
\end{document}